\documentclass{article}
\usepackage[utf8]{inputenc}

\title{Organized Notes}
\author{shiftehs }
\date{August 2015}

\usepackage{natbib}
\usepackage{graphicx}
\begin{document}
\maketitle

\section{Article 1: The Wisdom of the cell, Chien 2006}


Denne artikkelen er som en lærebok basert på omfattende studier forfatteren og gruppa hennes har gjort for å kartlegge hemodynamikkens utløsende rolle for mekanotranduksjon hos EC.
\\
\\ Først generelt om EC i blodårene. 
\\ EC liner blodårer. 
\\ Under følger en liste av normal funksjon av EC. 
\\1) EC fungerer som en permeabel barierre mellom blodet og blodåre veggen. 
\\2) EC migrerer, remodelerer seg selv(align with flow direction, se artikkel 2), prolifererer og undergo apoptosis. 
\\3) De regulerer kontraksjon av smooth muscle cells. 
\\4) De produserer og skiller ut stoffer i respons til mekanisk stimuli som fluid strekk og stress. 
\\5) EC modifiserere gen uttrykk som følge av mekanisk stimulig, for å regulere funksjonen sin.

\\
\\Mekanisk stress er:
\\ Kraft per enets areal, der eneheten er dyn per cm kvadrert
\\ Mekaniske krefter inkluderer circumferential stress soms kommer av trykket og shear stress som er parallelt til luminal overflaten på vessel veggen pga fluid flow. 

\\
\\Hvordan er mønsteret for stresses: 
\\Mønsteret varierer fra rett del av arterial treet, og forgerende deler. 
\\ For rette deler har stress og strech vel definerte retninger, og denne typen stress og strech induserer mekanismer som minimerer effekten av stress og strech som følge av trykk og flow, og opprettholder dermed homeostase. 
\\I forgrenede regioner og kurvede områder derimot, har ikke mekaniske stimulien en vel definert retning. Noe ikke utløser en feedback mekanisme for å minimere cellulær respons. Derimot til uønskede signaliserings prosesser. 

\\
\\ Hypotesen til denne artikkelen er: den har ingen. er en kartlegging av cellulære prosesser i EC.
Se egne notater. 

\section{Artikkel 2: Turbulent Fluid SS induserer vaskulær EC turnover in vitro,Davis 1986}

\\Dette var en eksperimentell (in vitro) studie, hovedpoengene er oppsummert: 
\\Denne studien viste at EC retter seg etter strømmen, når den er laminær, dvs. det skjer en endring i cytoskjelletet til EC slik at de antar en ellipsiodal morfologi (form) fra den opprinnelige polygonale formen(kontroll). 
Når EC blir utsatt for en svar* turbulent flow (TF) skjer ikke dette, de antar derimot en rund form og mange celler flyter opp av planet av monolaget med EC. 
Oppe fra monolaget entrer de celle cyklus(tror jeg, sjekk dette igjen).
\\
\\$*$ svak TF -  mulig karakteristikkene ved strømmen er det som utløser mekasnimen for celle deling heller enn strørrelsen på strømmen.
\\
\\En mulig tolkning er at monolaget hemmer vekst, pga kontakten med naboceller, og bare frigjøring fra monolaget kan stimulerer til vekst. 
(Husk når du var i laben så døde normale celler mediet når flasken ble full, mens kreftceller kunne vokse ukontrollert)
\\
\\Ved å bruke en av dem mest stabile isotopene til Hydrogen, tritium, $H^{3}$, viste eksperimenter at DNA syntese oftere fant sted hos celler utsatt for TF sammenlignet med LF. 
\\
\\Det er uvvist om det er såkalt 'cell refraction'(hva er det?, opp fra monolaget) som fører til EC turnover eller en annen hittil ukjent mekasniske som fører til EC turnover. 


\section{Artikkel 3: En Newtonsk viskøsistets model kan muligens overestimere effekten av WSS i ICA dome regionen og dermed underestimere rupture risiko, C
Meng 2011 }

I denne studien har forfatterne vurdert tre typer aneurysmer A) oblong, B) nesten sfæring og C) en bifurcation, og hvordan tre ulike viskositets modeller 1) Newtonsk, 2) Cassons og 3) HB vurderer tre kvantitative/ kvalitative egenskaper ved hymodynamikken 1) shear rate, 2) bloodets viskositet og 3) WSS. \\
\\Formålet var å test hvor godt en newtonsk modell fungrer for ICA aneurysmer.
Det er kjent at den er god for CFD simuleringen av store vessels, men hva med i i aneurysme sekken, hvor hemodynamikken muligens er annerledes. 
\\
\\Resultatet avar at den newtonske modellen ga høyere verdier for shear rate og WSS enn de ikke-newtonske modellene, og en lavere verdi på viskositet inni aneurysme sekken for oblonge aneurysmer med et komplekst strømnings mønster i den forstand at strømmen var delt inn i to vortekser inni aneurysme sekken. 
\\
\\Det blir referert til tidligere studier som viser til at lav WSS i aneurysme sekken er assosiert med ICA rupture, og blodpropp dannelse. Det påstås at det i slike regioner er en stagnant strøm og dermed mindre shear rate enn i moder/ parent vesselen. Vi kan derfor muligens anta at i slike lav WSS regioner vil den ikke newtonske effekten ikke være neglisjerbar, altså den bør muligens vektlegges i simuleringer. 
\\
\\En annen faktor det er mulig å merke seg fra denne artikkelen er at HB modellen gir høyere verdier for moder vesselen generelt og lavere WSS. 
\\
\\konklusjonen de trekker er at ikke-newtonske efekter spiller muligens en rolle inni aneurysme sekken og især når den er obling med flere vortekser


\\Kritikk/ spm til artikkelen: Er det kun gjort eksperimenter på tre aneurysmer? Er funnene statistik signifikante? 




\section{Artikkel 4: Sammenhengen mellom karakteristikker ved hemodynamikk og cerebral aneurysme ruture, Cebral 2011, del a)}

Dette var en CFD-studiet. Simuleringene bygde på pasient-spesifikke modeller av cerebrale aneurysmer. De ble konstruert ved at 3D bilder tatt av pasientene(Cat scan bilder kanskje) ble filterer for støy og delt opp vha en seeding algoritme(se notater fra Kent). Det ble lagd en overflate som det ble satt et grid over av tetrahedral elementer for de numeriske simuleringene. 
Resolusjon var på mellom 0.02cm - 0.01 cm(hva betyr det?).
Gridet bestod av 1-5 mill elementer.
\\
\\Blodstrømmen ble approksimert av unsteady 3D Navier-Stokes ligningene for en inkompressibel newtonsk væske og de antok at veggene var rigide. 
\\
\\De analyserte de hemodynamiske simularingene og fant fire kvalitative karakteristikker som knyttet rupture med hemodynamikk. Disse var følgende: 
\\1) Flow kompleksitet ble delt inn i a) simple versus b) complex.
En simple flow i aneurysmen betyr ar det kun er en resirkulasjons zone, mens en complex flow i aneurysmen betyr at det er flere resurkulajsons zoner og flere vortekser. 
\\2) Flow stabilitet ble delt inn i a) stabil vs b) ustabil. Der stabil betegner en flow som forblir uendret gjennom hjertesyklus,mens ustabil betyr endret flow gjennom hjertesyklus.  
\\3) Inflow Konsentrasjon ble delt inn i a) konsentrert inflow vs. b) diffus inflow. Konsentrerte strømmer også kalt jets fra moder vesselen penetrerer dypt inn i aneurysme sekken, mens b) diffuse ?? 
\\
\\4) Flow impingement zoner ble delt inn i a) store vs. b) små. Store flow impingement soner er områder der strømmen entrer aneurysme sekken og treffer veggen på en slik måte at den endrer retning. Disse områdene har høyt WSS. 
Små impingement soner skjer det et lite støt mot aneurysme veggen. 
Stor når området er 50 prosent av moder vesselen, og lite dersom under halvparten. 
\\
\\ Konklusonene forfatterne trakk i denne artikkelen var at kombinasjonene av konsentrert inflow, små impingement soner, et komplisert strømningsmønster og ustabil flow er korrelert til en klinisk fortid med rupture av aneurysme utposningen. 
\\
\\
\\Små til artikkelen : se egne notater. bla har jeg forstått de kvalitative egenskapenenes definisjoner rett? hva er konsenterert inflow? impingement soenr? Støtter dette en high flow teori da? 



\section{Artikkel 5: Kvantitative karakteristikker av de hemodynamiske miljøet hos ruptured og unruptured cerebral anaeurysms, 2011, del b)}

Dette er en videreføring av studien over, av de samme forfatterne. De vil nå definerer kvantitative metrikker kyttet til de allerede definerte kvalitative flow karakteristikkene definert i del a), og forholdet til aneurysme rupture. 
\\
\\De deler aneurysme forløp inn i tre stadier 1)genesis, 2) enlargement og 3) rupture. Rupture skjer når veggstress er større enn veggstyrke. Enlargement skjer når interaksjonene mellom hemodynamiksk load og mekanobiologisk respons resulterer i at veggen svekkes. 
Genesis, står det ikke noe om, men det kan kanskje forklares uti fra den første artikkelen. 
\\
hemodynamisk load => mekanobiologisk respongs => veggen svekkes* => enlargement av aneurysmen 
\\
\\$*$ det er to teorier for svekkelse av veggen, a) høy flow teori og en b) low flow teori. 
\\ I folge forskere som abonnerer på høy flow teorien mener de at høy WSS ødelegger EC og initeterer vegg remodellering og degenerasjon. Dvs blodes høye WSS felt er den drivende faktoren i utviklingen gav aneurysme geometri. 

\\law flow teoritikerne peker til lokalisert stagnasjon av blodstrømningn mot veggen på hodet av aneurysme som det som foråesaker dysfunksjon hos EC.

\section{Artikkel 6: Hemodynamiske- Morfologiske Discriminatns for ICA rupture, Xiang, 2011 }

Denne studien hadde som formål å finne signifikante morfologiske og hemodynamiske parametere for en ICA rupture vha angiografi og CFD.
De fire morfologiske parameterne de vurderte var størreøse, undulation index(UI), ellipticity index og nonspericity index. Mens de hemodynamiske parameterne var average WSS, max IC WSS, low WSS area, average oscillatory shear index, antall vortices og relativ resident time. 
Alle disse er definert i artikkelen som en metrikk.
Så brukte de multivariate logistic (?) regression analyse. 
Multivariate linear regression er en metode for at flere korrelerte avhengige variavle blir predikert. De brukte en backward seleksjons metode, dvs at de startet med alle kandidat variable i modellen, tester for deletion av hver variabel, utifra et kriterium, og sletter den variabelen som forbedrer modellen mest ved å bli eksludert. 
Resultatet var at den uavhengige variabelen ble size ratio(SR) i morfologi kategorien, og WSS og OSI i hemodynamikk kategorien. 
Og SR,WSS og OSI i den kombinerte modellen. 

Til slutt konkluderer de med at hemodynamikk er like viktig som morfologien, når det kommer til IA rupture, men at det må andre må se på det i en ennå større studiet som inkluderer oppfølging av pasienter med unruptured aneurysmer. 

\section{Artikkel 7: En studie av WSS i tolv aneurysmer mht. ulike viskositetsmodeller og flow betingelser, Kent 2013 }
\\
\\Denne artikkelen har som mål å rydde opp i modeller og flow metrikker for CFD simuleringer av aneurysmer, slik at CFD simuleringer ikke mister sin kredibilitet og the scientific miljøet kan nå en konsensus for flow metrikk og viskositets modell. 
\\
\\Vanlig bruk var Newtonsk væske, i nyere tid har mer kompliserte modeller blitt tatt i bruk. Disse har imidlertid relativt enkle bcs fra litteraturen og på et mindre antall aneurysmer. 
\\
\\De blir nevt at det er vist at EC vokser når det er lav WSS, men det er ikke nådd en konklusjon på om det er lav eller høy WSS som korrelerer med rupture. 
Dette tas ikke stilling til i akkurat denne artikkelen ? 
\\
\\Resultatene slik jeg tolker scatterplottene i fig 5, er at dataene settene fra de ulike vesikositetsmodellene og kontrollmodellen (newtonsk væske modellen) konsentrerer seg rundt en usynlig y = x linje. Generelt tror jeg man kan si at jo mer data settene er numerisk identiske, desto mer faller scattere på hverandre på identiteslinjen. 
\\
\\Litt usikker på hvordan jeg skal tolke fargene i plottet. 
Konlusjonen i studien er at de tre viskositets modellene og bcs viser sterk korrelasjon for tre ulike WSS metrikker. 
Hvordan set det fra fig 5 og 6? 

\section{ Artikkel 8: Om antagelsen om laminær strømning i fysiologiske flows, med bruk av cerebral aneurysme som et illustrerende eksempel, Kent, årstall uvisst}
\\
\\Et kapittel om at laminær flow hypotesen i patologier er utfordret, og fokuset er satt på turbulent og transisjons flow. 
Programmeringsmessig er laminær flow lettere å programmere, det tillater lav resolusjon, og stabiliesringsteknikker og temporal diskretisering med dissipasjon(varmetap). 
\\
\\Vanligvis har vi laminær flow for Re tall mindre enn 2 000, transisjons flow i område mellom 2300 - 4000 og turbulent flow når Re tallet er høyere enn 4 000. Men pga av pulserende strømning og avvik fra rettvinklet-rør-geometri så oppstår transisjon i blood vessels allerede for Re tall på 300. 
\\
\\Når det kommer til å modellere transisjons flow, kan man egentlig ikke bruke Navier-Stokes ligningene , selv om disse egner seg for fully developed turbulent flow, men det gjøres likevel. 
Problemer som da oppstår er å konstruerer en diskretisering uten dissipasjon, det trengs veldig høy oppløsning(resolusjon) sammenlignet med laminære simuleringer og bcs med ustabiliteter må tillates. 
\\
\\En benchmark studie av en gitt geometri utført av en rekke forskningsgrupper viste at simuleringer med koder med høy oppløsning gav høy flow instabilileter  sammenlignet med lav oppløsningskoder fra kommersielle agenter. 
\\Her følger noe om som jeg ikke helt forstod, men mulig de mener at det er problematisk at Kolmogorov lenge skalaen for blood flow simuleringer er av samme dimensjon som rød blod celle skalaen. Derfor bryter kontinuum hypotesen sammen. Hvorfor vet jeg ikke. 
\\
\\Et illustrerende eksempel: De har kontruert en modell av en aneurysme basert på en kanin anaurysme modell, for å studere overgangen til transisjons flow og terskelverdi. Det er brukt ca 3 mill celler, tidssteg er satt til ca 8e-6 sek, og blodet er modellert som en newtonsk væske. Det ble tidligere observert at overgangen til turbulent flow i store arterier skyldes geometrien heller en hjertets pulserende pumpemekanisme.Sagt på en annen måte, den turbulente karakteristikken ved flowen er tilstede allerede pga geometrien, pulseringen skrur den bare av og på. Dette skjer allerede ved Re tall mellom 200 - 300. 
Helt konkret, ble Navier-Stokes ligningene løst med IPCS med en CN metode for tid og det ikke-lineære leddet i det convection leddet i NS ligningene ble gjort lineært vha Picard(?). Dette ble kodet med P1 elementer for både hastigheten og trykket. 
\\
\\Les enkelte deler av denne igjen. 

\section{Artikkel 9: Algoritmisk derivasjon av adjoing og høy-nivå transient FE programmer, Funke, år uvisst}



\end{document}
